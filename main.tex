\documentclass{article}

\title{Relatório Final - IC}
\author{}

\usepackage[margin=0.8in]{geometry}
\usepackage[portuguese]{babel}
\usepackage{indentfirst}
\usepackage{fancyhdr}
\usepackage{tcolorbox} 
\usepackage{graphicx}
\usepackage{amsmath}
\usepackage{amssymb}
\usepackage{enumitem}
\usepackage{tabularx} % in the preamble
\usepackage{wrapfig}
\usepackage{subcaption}
\usepackage{caption}
\usepackage{listings}
\usepackage{tikz}
\usepackage{multirow}
\usepackage{cancel}
\usepackage{multicol}
\usepackage{float}
\usepackage{xcolor}

\usepackage{mathtools}  % loads »amsmath«
\newtagform{brackets}{[}{]}
\usetagform{brackets}

% Create a Todo list
\newlist{todolist}{itemize}{2}
\setlist[todolist]{label=$\square$}
\renewcommand{\figurename}{Figura}

% Fix vspace inside wrapfigure
\setlength\intextsep{0pt}

\newcolumntype{Y}{>{\centering\arraybackslash}X}

\captionsetup[lstlisting]{skip=10pt}

% Create a new command to be used in the align environment in multiple line equations do only the last equation is numbered  
\newcommand{\n}{\nonumber \\ }
\makeatletter
\let\inserttitle\@title
\makeatother
% Set the style of the page 
\pagestyle{fancy}
\fancyhf{}
\rhead{Relatório Final - IC \\ Pedro Henrique L. da Cruz \vspace{1px}}
\lhead{UNICAMP \\ Faculdade de Engenharia Mecânica\vspace{1px}}
\rfoot{Page \thepage}

\usepackage{hyperref}
\hypersetup{
    colorlinks=true,
    linkcolor=black,
    filecolor=file,
    urlcolor=black,
    citecolor=black
}

\usepackage{biblatex}
\addbibresource{biblography.bib}

\renewcommand*\contentsname{Sumário}
\lstdefinestyle{cStyle}{
    language=c,       % Linguagem para destacar CMake
    basicstyle=\ttfamily\footnotesize, % fonte monoespaçada menor
    keywordstyle=\color{blue}\bfseries,
    commentstyle=\color{gray}\itshape,
    stringstyle=\color{red},
    numbers=left,
    numberstyle=\tiny,
    stepnumber=1,
    numbersep=5pt,
    tabsize=4,
    breaklines=true,
    showstringspaces=false,
    frame=single,                 % coloca uma borda ao redor
    backgroundcolor=\color{gray!10}, % fundo levemente cinza
    captionpos=b
}

\lstdefinestyle{cmakeStyle}{
    language=sh,       % Linguagem para destacar CMake
    basicstyle=\ttfamily\footnotesize, % fonte monoespaçada menor
    keywordstyle=\color{blue}\bfseries,
    commentstyle=\color{gray}\itshape,
    stringstyle=\color{red},
    numbers=left,
    numberstyle=\tiny,
    stepnumber=1,
    numbersep=5pt,
    tabsize=4,
    breaklines=true,
    showstringspaces=false,
    frame=single,                 % coloca uma borda ao redor
    backgroundcolor=\color{gray!10}, % fundo levemente cinza
    captionpos=b
}

% Begin the Document 
\begin{document}

    \thispagestyle{empty}
    \begin{center}
    
        \begin{minipage}{0.12\textwidth}
            \includegraphics[width=1.5cm]{img/UNICAMP_logo.png} 
        \end{minipage}
        \begin{minipage}{0.50\textwidth}
            \begin{center}
                \textbf{Universidade Estadual de Campinas - UNICAMP} \\
                \textbf{Faculdade de Engenharia Mecânica - FEM}
            \end{center}
        \end{minipage}
        \begin{minipage}{0.15\textwidth}
            \hspace{15px}
            \includegraphics[width=1.5cm]{img/fem_logo.png} \\
        \end{minipage}
        
    \end{center}

% Title
\vspace{2cm}
\begin{center}
    \textbf{Utilização De Métodos de Tomada de Decisão Multi-Critério na Escolha de Controladores}
\end{center}

% Subtitle
\vspace{1cm}
\begin{center}
    \textbf{Relatório Final - Iniciação Científica}
\end{center}

% Supervisor and student
\vspace{2cm}
\begin{center}
    \begin{tabular}{l l}
        Orientador: & Prof. Dr. Tiago Henrique Machado \\
        E-mail: & tiagomh@fem.unicamp.br \\
        Aluno: & Pedro Henrique Limeira da Cruz \\
        E-mail: & p215663@dac.unicamp.br
    \end{tabular}
\end{center}

% Date
\vfill{}
\begin{center}
    Campinas, 06 de Dezembro de 2025.
\end{center}

\newpage
\tableofcontents
\newpage
\section{Introdução}
A escolha do controlador mais adequado para um sistema dinâmico depende de diversos critérios de desempenho que nem sempre apontam todos na mesma direção.
Aspectos como estabilidade, tempo de acomodação, sobre-sinal, erro em regime permanente, esforço de controle, robustez e até a complexidade de implementação influenciam de maneiras diferentes a
adequação de cada técnica de controle. Assim, a decisão não pode ser reduzida a um único indicador, exigindo uma visão holística que considere simultaneamente fatores técnicos e operacionais.

A análise multicritério surge justamente como uma abordagem capaz de organizar essa complexidade ao fornecer métodos estruturados para avaliar alternativas segundo múltiplos critérios. Em vez de
privilegiar um único indicador ou depender exclusivamente da intuição do projetista, os métodos de decisão multicritério permitem explicitar critérios relevantes, ponderar suas contribuições relativas
e revelar trade-offs que, muitas vezes, não são evidentes na avaliação isolada de cada indicador de desempenho. Com isso, o processo decisório torna-se mais transparente, reprodutível e melhor
fundamentado, especialmente em cenários onde diversas alternativas apresentam qualidades complementares.

A literatura de tomada de decisão multicritério mostra que diferentes métodos foram desenvolvidos para lidar com a natureza multidimensional e, muitas vezes, conflitante dos critérios envolvidos. Uma
distinção fundamental é entre métodos compensatórios, nos quais um bom desempenho em um critério pode compensar um desempenho inferior em outro, e métodos não compensatórios, que impõem limites
rígidos e evitam compensações excessivas \cite{belton2002mcda,triantaphyllou2000mcdm}. Outra categoria amplamente discutida é a dos métodos de ponderação objetiva, que atribuem
pesos com base em propriedades estatísticas dos dados, como variabilidade e contraste entre critérios, sendo o método CRITIC um dos exemplos mais conhecidos \cite{diakoulaki1995critic}. Métodos
outranking, como ELECTRE e PROMETHEE, também desempenham papel importante ao permitir comparações parciais entre alternativas por meio de relações de dominância
\cite{roy1991electre,brans1985promethee}. Além disso, abordagens baseadas em programação multiobjetivo oferecem uma formulação matemática mais ampla para problemas com múltiplos objetivos conflitantes
\cite{miettinen1999multiobjective}.

Considerando esse contexto, propõe-se neste trabalho a aplicação de métodos de decisão multicritério para apoiar a escolha do controlador mais adequado para um sistema CNC de dois eixos. Para isso,
utilizam-se os métodos CRITIC, para determinação objetiva dos pesos dos critérios de desempenho, e TOPSIS, para classificação das alternativas com base na proximidade da solução ideal, permitindo
avaliar controladores distintos de forma sistemática e quantitativa. A análise é conduzida a partir de um modelo dinâmico dos eixos X e Y desenvolvido em Simulink, no qual diversos critérios são
extraídos, possibilitando uma comparação abrangente entre as opções de controladores estudadas.


\section{Objetivos}
Propõe-se neste trabalho a análise multicriterial dos controladores candidatos para a CNC utilizando os métodos CRITIC e TOPSIS, de forma a determinar, de maneira quantitativa e objetiva, qual
controlador apresenta o melhor desempenho global para o sistema dinâmico de dois eixos. Para tal, serão realizadas as seguintes etapas:

\begin{itemize}
    \item A modelagem em Simulink do sistema dinâmico da máquina CNC.
    \item A obtenção dos critérios de desempenho dos controladores avaliados, incluindo 14 métricas (7 para o eixo X e 7 para o eixo Y).
    \item A análise dos critérios de resposta ao degrau, tais como Rise Time, Transient Time, Settling Time, Overshoot e Peak Time.
    \item A avaliação de critérios relacionados ao seguimento de trajetória, como Integral Absolute Error (IAE) e Integral no Tempo da Tensão (ITV).
    \item A aplicação do método CRITIC para determinação dos pesos objetivos dos critérios, considerando suas correlações e contrastes.
    \item A utilização do método TOPSIS para ranqueamento das alternativas de controladores, identificando qual delas se aproxima mais da solução ideal e melhor atende às exigências do sistema.
\end{itemize}

\newpage
\section{Metodologia e Simulações}
Nesta seção são apresentadas as etapas necessárias para identificar o sistema a partir da sua resposta ao degrau, o modelo simulink usado de base para as simulações e o tuning dos controladores PI,
PID e LQR.

\subsection{Identificação e Modelagem Do Sistema}
A fim de obtermos um modelo dinâmico do sistema, foi utilizada a resposta ao degrau obtida por Martin Jurek \cite{JurekWagnerova2019} do eixo x (belt + massa) de uma CNC de dois eixos e o processo de identificação de um sistema de 
segunda ordem a partir da sua resposta descrita em \cite{Busoniu2018}, que tem a função de transferência descrita pela equação \ref{eq:sec_order_sys}.
\begin{align}
    H(s) = K\frac{\omega_n^2}{s^2 + 2\xi\omega_n s + \omega_n^2 },
    \label{eq:sec_order_sys}
\end{align}
onde: 
\begin{itemize}
    \item $K$: Representa o ganho estático do sistema;
    \item $\omega_n$: Representa a frequência natural do sistema;
    \item $\xi$: Representa a razão de amortecimento do sistema.
\end{itemize}

O primeiro passo para a identificação é a determinar o valor em estado estacionário posterior a todas as oscilações, denominado de $y_{ss}$, sendo igual ao ganho $K$ para sistemas partindo do zero:
\begin{align}
    y_{ss} = K = 0.05 \times 10^{-3}.
\end{align}

A partir desse valor, é possível determinar o máximo overshoot $M$:
\begin{align}
    M = \frac{y_{pico} - y_{ss}}{y_{ss}},
\end{align}
que por sua vez  é utilizado para determinar o fator de amortecimento:

\begin{align}
    \xi = \frac{\log 1/M}{\sqrt{\pi^2 + \log^2 M}} =  0.4384.
\end{align}

Com isso, em conjunto com o período entre oscilações $T_0$, obtemos o valor da frequência natural para o eixo $x$:
\begin{align}
    \omega_n = \frac{2\pi}{T_0\sqrt{1 - \xi^2}} = 699.09
\end{align}

Já para a modelagem do motor, foi utilizado o modelo de um motor DC \cite{CTMS_MotorSpeed_SystemModeling}, dada pela equação \ref{eq:mod_motor}, sendo os 
parâmetros obtidos a partir do datasheet de um motor comercial \cite{Foneacc_37GB-555}.

\begin{align}
    P(s) = \frac{K}{(Js+b)(Ls+R) + K^2},
    \label{eq:mod_motor}
\end{align}
onde:

\begin{itemize}
    \item $K$: Representa a constante de torque e emf;
    \item $J$: Momento de inércia do rotor;
    \item $b$: Constante de viscosidade;
    \item $R$: Resistência elétrica do estator;
    \item $L$: Indutância elétrica do estator.
\end{itemize}


\clearpage
\subsection{Controlador PI}
Após implementar o conjunto de equações de transferências que compõe o sistema motor, belt e massa para cada eixo em subsistemas do simulink (dada pela figura \ref{fig:simulink_massa_belt_motor}), onde tem como 
entrada a saída do controlador $PI$, foram obtidos os valores $K_p = 1781.269$ e $K_i=0.094$ para o eixo X, e $K_p=1859.321$ e $K_i=0.098$ para o eixo Y, a partir do tuning baseado em função de transferência incluido no \emph{control toolbox} do
MATLAB.

\vspace{10px}
\begin{figure}[h!]
    \centering
    \includegraphics[width=0.8\textwidth]{img/simulink_motor_belt_massa.png}
    \caption{Simulink - Subsistema Motor, Massa e Belt}
    \label{fig:simulink_massa_belt_motor}
\end{figure}
\vspace{10px}

A partir dessa configuração do sistema, foram feitos dois testes e coletados os dados de posição esperada, posição real e esforço do atuador tanto para o eixo X quanto para o eixo
Y para a resposta ao degrau e perfil trapezoidal  de velocidade em uma trajetória diagonal para cada um dos eixos, que podem ser observados nas figuras \ref{fig:pi_x_step} e \ref{fig:pi_y_step} para a resposta 
ao degrau e \ref{fig:pi_x_diag} e \ref{fig:pi_y_diag} para resposta a uma trajetória diagonal.


\begin{figure}[h!]
    \centering
    \begin{subfigure}[b]{0.47\textwidth}
        \centering
        \includegraphics[width=\textwidth]{img/pi_diag_x.jpg}
        \caption{Eixo X}
        \label{fig:pi_x_diag}
    \end{subfigure}
    \hfill
    \begin{subfigure}[b]{0.47\textwidth}
        \centering
        \includegraphics[width=\textwidth]{img/pi_diag_y.jpg}
        \caption{Eixo Y}
        \label{fig:pi_y_diag}
    \end{subfigure}
    \caption{Controlador PI - Perfil Trapezoidal}
    \label{fig:pi_diag}
\end{figure}

\begin{figure}[h!]
    \centering
    \begin{subfigure}[b]{0.47\textwidth}
        \centering
        \includegraphics[width=\textwidth]{img/pi_step_x.jpg}
        \caption{Eixo X}
        \label{fig:pi_x_step}
    \end{subfigure}
    \hfill
    \begin{subfigure}[b]{0.48\textwidth}
        \centering
        \includegraphics[width=\textwidth]{img/pi_step_y.jpg}
        \caption{Eixo Y}
        \label{fig:pi_y_step}
    \end{subfigure}
    \caption{Controlador PI - Resposta ao Degrau}
    \label{fig:pi_step}
\end{figure}


\clearpage
\subsection{Controlador PID}
De forma análogo a como foi realizado para o controlador PI, foi utilizado a função de tuning baseado 
em função de transferência do \emph{control toolbox} para definir os valores dos ganhos dos controladores PID, resultando nos valores 
apresentados na tabela \ref{tab:pid_gains}.

\begin{table}[h!]
    \centering
    \begin{tabular}{lcc}
        \hline
        \hline
        &Eixo X&Eixo Y\\
        \hline
        $K_p$&1216.55&1235.48\\
        $K_i$&0.71&0.60\\
        $K_d$&0.08&0.05\\
        \hline
        \hline
    \end{tabular}
    \caption{Ganhos - Controlador PID}
    \label{tab:pid_gains}
\end{table}
\vspace{10px}

A partir desses valores, foram realizados os testes de resposta ao degrau e diagonal, como demonstrado pelas figuras \ref{fig:pid_diag} e \ref{fig:pid_step}.

\vspace{10px}
\begin{figure}[h!]
    \centering
    \begin{subfigure}[b]{0.47\textwidth}
        \centering
        \includegraphics[width=\textwidth]{img/pid_diag_x.jpg}
        \caption{Eixo X}
        \label{fig:pid_x_diag}
    \end{subfigure}
    \hfill
    \begin{subfigure}[b]{0.47\textwidth}
        \centering
        \includegraphics[width=\textwidth]{img/pid_diag_y.jpg}
        \caption{Eixo Y}
        \label{fig:pid_y_diag}
    \end{subfigure}
    \caption{Controlador PID - Perfil Trapezoidal}
    \label{fig:pid_diag}
\end{figure}

\vspace{10px}
\begin{figure}[h!]
    \centering
    \begin{subfigure}[b]{0.47\textwidth}
        \centering
        \includegraphics[width=\textwidth]{img/pid_step_x.jpg}
        \caption{Eixo X}
        \label{fig:pid_x_step}
    \end{subfigure}
    \hfill
    \begin{subfigure}[b]{0.48\textwidth}
        \centering
        \includegraphics[width=\textwidth]{img/pid_step_y.jpg}
        \caption{Eixo Y}
        \label{fig:pid_y_step}
    \end{subfigure}
    \caption{Controlador PID - Resposta ao Degrau}
    \label{fig:pid_step}
\end{figure}








\subsection{Controlador LQR}
\subsection{Método CRITIC}
\subsection{Método TOPSIS}

\section{Conclusão}

\newpage
\printbibliography[title={Referências}]

\end{document}
\documentclass{article}

\title{Relatório Final - IC}
\author{}

\usepackage[margin=0.8in]{geometry}
\usepackage[portuguese]{babel}
\usepackage{indentfirst}
\usepackage{fancyhdr}
\usepackage{tcolorbox} 
\usepackage{graphicx}
\usepackage{amsmath}
\usepackage{amssymb}
\usepackage{enumitem}
\usepackage{tabularx} % in the preamble
\usepackage{wrapfig}
\usepackage{subcaption}
\usepackage{caption}
\usepackage{listings}
\usepackage{tikz}
\usepackage{multirow}
\usepackage{cancel}
\usepackage{multicol}
\usepackage{float}
\usepackage{xcolor}

\usepackage{mathtools}  % loads »amsmath«
\newtagform{brackets}{[}{]}
\usetagform{brackets}

% Create a Todo list
\newlist{todolist}{itemize}{2}
\setlist[todolist]{label=$\square$}
\renewcommand{\figurename}{Figura}

% Fix vspace inside wrapfigure
\setlength\intextsep{0pt}

\newcolumntype{Y}{>{\centering\arraybackslash}X}

\captionsetup[lstlisting]{skip=10pt}

% Create a new command to be used in the align environment in multiple line equations do only the last equation is numbered  
\newcommand{\n}{\nonumber \\ }
\makeatletter
\let\inserttitle\@title
\makeatother
% Set the style of the page 
\pagestyle{fancy}
\fancyhf{}
\rhead{Relatório Final - IC \\ Pedro Henrique L. da Cruz \vspace{1px}}
\lhead{UNICAMP \\ Faculdade de Engenharia Mecânica\vspace{1px}}
\rfoot{Page \thepage}

\usepackage{hyperref}
\hypersetup{
    colorlinks=true,
    linkcolor=black,
    filecolor=file,
    urlcolor=black,
    citecolor=black
}

\usepackage{biblatex}
\addbibresource{biblography.bib}

\renewcommand*\contentsname{Sumário}
\lstdefinestyle{cStyle}{
    language=c,       % Linguagem para destacar CMake
    basicstyle=\ttfamily\footnotesize, % fonte monoespaçada menor
    keywordstyle=\color{blue}\bfseries,
    commentstyle=\color{gray}\itshape,
    stringstyle=\color{red},
    numbers=left,
    numberstyle=\tiny,
    stepnumber=1,
    numbersep=5pt,
    tabsize=4,
    breaklines=true,
    showstringspaces=false,
    frame=single,                 % coloca uma borda ao redor
    backgroundcolor=\color{gray!10}, % fundo levemente cinza
    captionpos=b
}

\lstdefinestyle{cmakeStyle}{
    language=sh,       % Linguagem para destacar CMake
    basicstyle=\ttfamily\footnotesize, % fonte monoespaçada menor
    keywordstyle=\color{blue}\bfseries,
    commentstyle=\color{gray}\itshape,
    stringstyle=\color{red},
    numbers=left,
    numberstyle=\tiny,
    stepnumber=1,
    numbersep=5pt,
    tabsize=4,
    breaklines=true,
    showstringspaces=false,
    frame=single,                 % coloca uma borda ao redor
    backgroundcolor=\color{gray!10}, % fundo levemente cinza
    captionpos=b
}

% Begin the Document 
\begin{document}

    \thispagestyle{empty}
    \begin{center}
    
        \begin{minipage}{0.12\textwidth}
            \includegraphics[width=1.5cm]{img/UNICAMP_logo.png} 
        \end{minipage}
        \begin{minipage}{0.50\textwidth}
            \begin{center}
                \textbf{Universidade Estadual de Campinas - UNICAMP} \\
                \textbf{Faculdade de Engenharia Mecânica - FEM}
            \end{center}
        \end{minipage}
        \begin{minipage}{0.15\textwidth}
            \hspace{15px}
            \includegraphics[width=1.5cm]{img/fem_logo.png} \\
        \end{minipage}
        
    \end{center}

% Title
\vspace{2cm}
\begin{center}
    \textbf{Utilização De Métodos de Tomada de Decisão Multi-Critério na Escolha de Controladores}
\end{center}

% Subtitle
\vspace{1cm}
\begin{center}
    \textbf{Relatório Final - Iniciação Científica}
\end{center}

% Supervisor and student
\vspace{2cm}
\begin{center}
    \begin{tabular}{l l}
        Orientador: & Prof. Dr. Tiago Henrique Machado \\
        E-mail: & tiagomh@fem.unicamp.br \\
        Aluno: & Pedro Henrique Limeira da Cruz \\
        E-mail: & p215663@dac.unicamp.br
    \end{tabular}
\end{center}

% Date
\vfill{}
\begin{center}
    Campinas, 06 de Dezembro de 2025.
\end{center}

\newpage
\tableofcontents
\newpage
\section{Introdução}
A escolha do controlador mais adequado para um sistema dinâmico depende de diversos critérios de desempenho que nem sempre apontam todos na mesma direção.
Aspectos como estabilidade, tempo de acomodação, sobre-sinal, erro em regime permanente, esforço de controle, robustez e até a complexidade de implementação influenciam de maneiras diferentes a
adequação de cada técnica de controle. Assim, a decisão não pode ser reduzida a um único indicador, exigindo uma visão holística que considere simultaneamente fatores técnicos e operacionais.

A análise multicritério surge justamente como uma abordagem capaz de organizar essa complexidade ao fornecer métodos estruturados para avaliar alternativas segundo múltiplos critérios. Em vez de
privilegiar um único indicador ou depender exclusivamente da intuição do projetista, os métodos de decisão multicritério permitem explicitar critérios relevantes, ponderar suas contribuições relativas
e revelar trade-offs que, muitas vezes, não são evidentes na avaliação isolada de cada indicador de desempenho. Com isso, o processo decisório torna-se mais transparente, reprodutível e melhor
fundamentado, especialmente em cenários onde diversas alternativas apresentam qualidades complementares.

A literatura de tomada de decisão multicritério mostra que diferentes métodos foram desenvolvidos para lidar com a natureza multidimensional e, muitas vezes, conflitante dos critérios envolvidos. Uma
distinção fundamental é entre métodos compensatórios, nos quais um bom desempenho em um critério pode compensar um desempenho inferior em outro, e métodos não compensatórios, que impõem limites
rígidos e evitam compensações excessivas \cite{belton2002mcda,triantaphyllou2000mcdm}. Outra categoria amplamente discutida é a dos métodos de ponderação objetiva, que atribuem
pesos com base em propriedades estatísticas dos dados, como variabilidade e contraste entre critérios, sendo o método CRITIC um dos exemplos mais conhecidos \cite{diakoulaki1995critic}. Métodos
outranking, como ELECTRE e PROMETHEE, também desempenham papel importante ao permitir comparações parciais entre alternativas por meio de relações de dominância
\cite{roy1991electre,brans1985promethee}. Além disso, abordagens baseadas em programação multiobjetivo oferecem uma formulação matemática mais ampla para problemas com múltiplos objetivos conflitantes
\cite{miettinen1999multiobjective}.



\section{Objetivo}


\section{Metodologia}
\section{Conclusão}

\newpage
\printbibliography[title={Referências}]

\end{document}